\subsubsection{Test der Anwendung}
Für das Testen von der Map-Reduce-Implementierung wird das MRUnit-Framework von Apache
\cite{mrunit} genutzt. Damit lassen sich komfortabel Unit-Test für die map- und reduce-Funktionen
schreiben. Allerdings gestaltet sich das Testen von Map-Reduce-Implementierungen mit
der Hbase-API schwierig. Im Rahmen des Projektes ist es nicht möglich, die Ausgabe der
reduce-Funktion zu überprüfen. Das Problem liegt bei der Angabe des Datentyps für
die Ausgabe der reduce-Funktion: während die Klasse \texttt{TableReducer} als reduce-Ausgabe
den Datentyp \texttt{Mutation} festlegt, wird damit der Zugang zur reduce-Ausgabe auf diesen
Datentyp beschränkt. Dass der Mutation-Datentyp nur eine Schnittstelle ist und zur
Laufzeit ein Put-Objekt verwendet wird, lies sich im Test nicht beschreiben.
Damit allerdings bleibt der Zugang auf das Ergebnis der reduce-Funktion auf den
Mutation-Datentyp beschränkt. Eine ordentliche Überprüfung des Ergebnisses konnte damit
nicht durchgeführt werden. Damit bleiben die Unit-Test im Hbase-Fall auf die map-Funktion
beschränkt. 

Alle weiteren Test, also Integrations- und Systemtests wurden manuell durchgeführt.
Der Aufbau eines Test-Frameworks für diese Tests würde den Rahmen des Projektes 
überschreiten. Auch auf Oberflächentest, zum Beispiel mittels Selenium, wurde verzichtet
und manuell durchgeführt.

\subsubsection{Performance-Aspekte}
TODO


\subsubsection{Implementierung von MapReduce-Funktionen}

Dieser Abschnitt beschreibt die Implementierung von Map-Reduce-Funktionen, die auf den Daten
des Million-Song-Datensatzes arbeiten. Vorausgesetzt wird, dass der Million-Song-Datensatz als
CSV-Datei bereit in das HDFS-Dateisystem importiert ist. Die hier vorgestellten Map-Reduce-Funktionen
arbeiten ausschließlich mit Daten auf dem HDFS-Dateisystem, abgesehen von den Zwischenergebnissen,
die auf dem lokalen Dateisystem des jeweiligen Knotens abgelegt werden.
Für die Implementierung wird die Java-API des Map-Reduce von Hadoop verwendet. Wichtig ist,
dass die aktuelle mapreduce-API verwendet wird, und nicht die inzwischen veraltete mapred-API.
Eine gute Hilfe bei der Entwicklung von map- und reduce-Funktionen bietet \cite{miner2012mapreduce}.

Die erste Implementierung der Map-Reduce-Funktionen \ref{mrCompleteToStripped} überführt alle Songs aus der ursprünglichen
CSV-Datei in eine neue CSV-Datei, in der erstens kein Song mehr doppelt vorkommt und zweitens
ein Song nur noch eine Untermenge der ursprünglichen Attribute enthält. Damit soll der ursprüngliche
Datenbestand auf einen reduzierten Datenbestand mit den für den Nutzer interessanten Attributen 
transformiert werden. Die map-Funktion ist dafür verantwortlich, die Attribute eines Musikstückes
zu identifizieren und daraus mit einen neuen Musikstück-Eintrag zu erzeugen, der nur noch die interessanten
Attribute enthält.
Zuerst muss definiert werden, welche Eingabeparameter die map-Funktion bekommt. Dazu gibt es verschiedene
Eingabeformate der Map-Reduce-API, unter denen man wählen kann. Weil die Daten in der CSV-Datei für das
Map-Reduce-Framework nichts weiter als reiner Text ist, wird das \textit{Text}-Format als Eingabe für
die map-Funktion gewählt, welches gleichzeitig auch das Standard-Format für die map-Funktion ist.
Das Format des Schlüssel-Parameters wird nicht weiter angegeben, weil der Schlüsselwert in diesem Falle 
während der Verarbeitung keine Rolle spielt.

Die map-Funktion teilt den übergebenen Wert des Textes in die Song-Attribute auf, indem es den Text
als einzelnen String betrachtet und mittels der \textit{split}-Methode ein Feld von Strings erzeugt, die durch
das Komma im ursprünglichem String getrennt sind. Dadurch sind die Musik-Attribute als Strings nun einzelnen verfügbar.
Der neue Eintrag für den Song wird durch das Erzeugen eines neuen Strings und dem Anhängen ausgewählter
Attribute erzeugt. Ist der neue String fertig zusammen gebaut, wird er wieder in das hadoop-spezifische \textit{Text}
konvertiert und als Zwischenergebnis dem Map-Reduce-Framework übergeben. Dieses erwartet neben dem 
Wert aber auch einen Schlüssel. Der Schlüssel ist in dem Falle ebenfalls vom Typ \textit{Text}, der als Wert die
Song-ID des Musikstückes enthält.

Die reduce-Funktion bekommt beim Aufruf als Eingabe eine Liste jener Zwischenergebnisse, die den gleichen 
Schlüssel-Wert haben, was in diesem Falle die Song-ID ist. 
Diese Funktion schreibt nun als Ergebnis das erste Element der Liste raus. Alle anderen
Einträge werden ignoriert. Somit erzeugt diese Map-Reduce-Implementierung eine neue Liste
von Musikstücken ohne Mehrfacheinträge für den gleichen Musiksong.

Die nächste Map-Reduce-Implementierung  ermöglicht die Ermittlung, wie viele 
Musikstücke ein bestimmter Künstler bereits geschrieben beziehungsweise herausgegeben
hat \ref{mrCountArtistSongs}. Diese Implementierung arbeitet dabei auf den Datenstrukturen von Hbase und nicht,
wie zuvor, direkt auf dem HDFS. Dies drückt sich vor allem in der Verwendung der 
\texttt{TableMapper}- und \texttt{TableReducer}-Klassen aus, die von Hbase als spezialisierte \texttt{Mapper}- beziehungsweise \texttt{Reducer}-Klassen zur Verfügung gestellt werden.
Diese spezialisierten Klassen sind in der Lage, das Ergebnis einer Hbase-Abfrage so zu
zerteilen, dass das Map-Reduce-Framework der map-Funktion immer eine Datenreihe 
als Eingabe übergibt. Ähnliches gilt auch für die reduce-Funktion, dessen Ergebnis
nun nicht direkt eine Datei auf dem HDFS ist, sondern die Daten mit einem Hbase-Aufruf übergeben
werden.

Die map-Funktion überführt die Daten einer ganzen
Datenreihe aus der Hbase-Datenbank in ein Schlüssel-Wert-Paar. Dieses enthält, ähnlich 
dem Beispiel mit dem Zählen von Wörtern, als Schlüssel den Künstlernamen und als Wert
die Zahl $1$. Damit repräsentiert das Zwischenergebnis einen Song eines bestimmten Künstlers.

Die reduce-Funktion bekommt die Liste aller
Zwischenergebnisse des gleichen Künstlers und zählt die Elemente dieser Liste. Das Ergebnis
repräsentiert die Anzahl der Songs von diesem Künstler, welches als Ergebnis in eine 
Hbase-Tabelle geschrieben wird.


Die letzte Map-Reduce-Implementierung realisiert die Anforderung, dass die Musikstücke über
ihren Titel auf ein bestimmtes Thema hin untersucht werden \ref{mrCountSongsWithTopic}. Der Titel muss dafür ein bestimmtes
Wort, dass das Thema festlegt, beinhalten. Als Ergebnis soll eine Liste von Künstlern erstellt werden,
die jeweils alle mindestens ein Musikstück produziert haben, dass das Thema enthält sowie auch
die Anzahl, wie viele Musikstücke mit dem jeweiligen Thema vorhanden sind.

Die map-Funktion filtert die Musikstücke danach, ob sie im Title ein bestimmtes Wort enthalten.
Ist dies der Fall, wir als Schlüssel-Wert-Paar der Künstlername als Schlüssel und die $1$ als 
Wert als Zwischenergebnis geschrieben. Enthält das Musikstück das gesuchte Wort nicht,
so kehrt die Funktion ohne Zwischenergebnis zurück.

Die reduce-Funktion zählt, wie in der vorherigen Map-Reduce-Implementierung auch, die Elemente der Liste des gleichen Künstlers.
Das Ergebnis wird ebenfalls in eine Hbase-Tabelle geschrieben und repräsentiert die Anzahl der Musikstücke, die der
Künstler zu dem bestimmten Thema geschrieben hat.

\subsubsection{Abstürze bei der Ausführung}
Bei großen Datenmengen, wie es insbesondere bei der ersten Map-Reduce-Implementierung \ref{mrCompleteToStripped} der Fall ist,
führt die Ausführung des Map-Reduce-Jobs zum Absturz einiger bis alle Programme auf einem oder mehreren Rechner-Knoten
, die vom gleichen User (team6) gestartet wurden.
In den Logs ist ersichtlich, dass beispielsweise das HDFS ein Kill-Signal bekommt und damit die Aufforderung, sich zu beenden.
Wer das Kill-Signal sendet, und vor allem warum, konnte bislang nicht ermittelt werden. Zumindest weder in den Logs der 
Hadoop-Programme noch im Kernel-Log konnte ein entsprechender Eintrag gefunden werden, der auf ein Senden eines
Kill-Signales hinweist. Denkbar wäre beispielsweise eine zu große Nutzung von Speicher, sodass das Betriebssystem ein
oder mehrere Prozesse zum eigenem Schutz beendet. Entsprechende Einträge wurden aber nicht gefunden.

Recherchen im Internet legen aber weiterhin den Verdacht nahe, dass eine ausufernde Speicher-Anforderung von den Java-Prozessen
an das Betriebssystem ein Problem sein könnte. Im Zuge dessen wurde die Speichernutzung von Map-Reduce manuell konfiguriert 
(siehe \ref{anforderung:berechungSpeicher}). Dies brachte jedoch nicht den erhofften Erfolg.

Erst die Aktivierung der Kompression von den Zwischenergebnissen der Map-Prozesse brachte eine solche Verbesserung, dass zumindest
die zweite \ref{mrCountArtistSongs} und dritte \ref{mrCountSongsWithTopic} weitestgehend ohne Abstürze ausführbar waren. Die Kompression
lässt sich über den \texttt{apreduce.map.output.compress} in der \textit{mapred-site.xml} auf \textit{true} stellen.
Der Erfolg dieser Maßnahme leitet zu einer weiteren Annahme, dass eventuell die Netzwerklast zu den Abstürzen führen
könnte. Dies lies sich aber durch Beobachtungen der Bandbreite eine Rechner-Knotens während der Ausführung
eine Map-Reduce-Jobs nicht bestätigen.

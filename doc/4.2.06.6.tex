\subsubsection{Test der Anwendung}

Für das Testen von der Map-Reduce-Implementierung wird das MRUnit-Framework von Apache
\cite{mrunit} genutzt. Damit lassen sich komfortabel Unit-Test für die map- und reduce-Funktionen
schreiben. Allerdings gestaltet sich das Testen von Map-Reduce-Implementierungen mit
der Hbase-API schwierig. Im Rahmen des Projektes ist es nicht möglich, die Ausgabe der
reduce-Funktion zu überprüfen. Das Problem liegt bei der Angabe des Datentyps für
die Ausgabe der reduce-Funktion: während die Klasse \texttt{TableReducer} als reduce-Ausgabe
den Datentyp \texttt{Mutation} festlegt, wird damit der Zugang zur reduce-Ausgabe auf diesen
Datentyp beschränkt. Dass der Mutation-Datentyp nur eine Schnittstelle ist und zur
Laufzeit ein Put-Objekt verwendet wird, lies sich im Test nicht beschreiben.
Damit allerdings bleibt der Zugang auf das Ergebnis der reduce-Funktion auf den
Mutation-Datentyp beschränkt. Eine ordentliche Überprüfung des Ergebnisses konnte damit
nicht durchgeführt werden. Damit bleiben die Unit-Test im Hbase-Fall auf die map-Funktion
beschränkt. 

Alle weiteren Test, also Integrations- und Systemtests wurden manuell durchgeführt.
Der Aufbau eines Test-Frameworks für diese Tests würde den Rahmen des Projektes 
überschreiten. Auch auf Oberflächentest, zum Beispiel mittels Selenium, wurde verzichtet
und manuell durchgeführt.

Bei großen Datenmengen, wie es insbesondere bei der ersten Map-Reduce-Implementierung \ref{mrCompleteToStripped} der Fall ist,
führt die Ausführung des Map-Reduce-Jobs zum Absturz einiger bis alle Programme auf einem oder mehreren Rechner-Knoten
, die vom gleichen User (team6) gestartet wurden.
In den Logs ist ersichtlich, dass beispielsweise das HDFS ein Kill-Signal bekommt und damit die Aufforderung, sich zu beenden.
Wer das Kill-Signal sendet, und vor allem warum, konnte bislang nicht ermittelt werden. Zumindest weder in den Logs der 
Hadoop-Programme noch im Kernel-Log konnte ein entsprechender Eintrag gefunden werden, der auf ein Senden eines
Kill-Signales hinweist. Denkbar wäre beispielsweise eine zu große Nutzung von Speicher, sodass das Betriebssystem ein
oder mehrere Prozesse zum eigenem Schutz beendet. Entsprechende Einträge wurden aber nicht gefunden.

Recherchen im Internet legen aber weiterhin den Verdacht nahe, dass eine ausufernde Speicher-Anforderung von den Java-Prozessen
an das Betriebssystem ein Problem sein könnte. Im Zuge dessen wurde die Speichernutzung von Map-Reduce manuell konfiguriert 
(siehe \ref{anforderung:berechungSpeicher}). Dies brachte jedoch nicht den erhofften Erfolg.

Erst die Aktivierung der Kompression von den Zwischenergebnissen der Map-Prozesse brachte eine solche Verbesserung, dass zumindest
die zweite \ref{mrCountArtistSongs} und dritte \ref{mrCountSongsWithTopic} weitestgehend ohne Abstürze ausführbar waren. Die Kompression
lässt sich über den \texttt{apreduce.map.output.compress} in der \textit{mapred-site.xml} auf \textit{true} stellen.
Der Erfolg dieser Maßnahme leitet zu einer weiteren Annahme, dass eventuell die Netzwerklast zu den Abstürzen führen
könnte. Dies lies sich aber durch Beobachtungen der Bandbreite eine Rechner-Knotens während der Ausführung
eine Map-Reduce-Jobs nicht bestätigen.

Um die Leistungsfähigkeit des Systems zu testen, wurde der vorgegebene Anwendungsfall zur Suche nach Songs eines bestimmten
Themas als Map-Reduce-Job implementiert \ref{mrCountSongsWithTopic}.

Ein Auszug des Ergebnisses des Performance-Test in der Konsole zeigt \ref{performance:console} (vollständige Ausgabe im Anhang \ref{performanceLog}).
Es zeigt insbesondere die Start- und End-Zeit des Map-Reduce-Jobs. Demnach benötigte der Job 
$29$ Sekunden bis zum Ergebnis (exakt: $29510$ ms). Insgesamt lieferte HBase ca $2,2$ Gigabyte an Daten als Ergebnis 
für die map-Phase zurück. Dabei wurden genau $10.000$ Datenreihen beziehungsweise Musikstücke durchsucht,
was erwartungsgemäß die gesamte Menge an gespeicherten Musikstücken entspricht. 
Das Ergebnis der map-Phase bestand aus $368$ Zwischen-Ergebnissen, das bedeutet in $368$ Musikstücke wurde
der Text \textit{love} im Titel gefunden. Die reduce-Phase reduziert das Ergebnis nochmal auf $329$ Musiksongs, was bedeutet,
dass einige Künstler mehrere Songs mit dem Suchstring im Titel besitzen.

\begin{figure}
	\centering
	\includegraphics[width=1.0\textwidth]{images/{06.performanceConsole}.png}
	\caption{Screenshot von der Konsolen-Ausgabe des Leistungs-Tests (Ausschnitt)}
	\label{performance:console}
\end{figure}

Der Screenshot \ref{performance:yarn} zeigt den Leistungs-Test auf der YARN-Weboberfläche.

\begin{figure}
	\centering
	\includegraphics[width=1.0\textwidth]{images/{06.performanceYarn}.png}
	\caption{Screenshot von der YARN-Weboberfläche mit dem Leistungs-Tests (Ausschnitt)}
	\label{performance:yarn}
\end{figure}

\subsubsection{Schnittstelle der Anwendung zur Datenbank}\label{schnittstelle}

Damit die Client-Anwendung die Daten aus HBase bekommen kann, muss eine Verbindung zur Datenbank aufgebaut werden.
Das erreicht man indem man ein \textit{Configuration}-Objekt erstellt und ihm alle benötigten Daten zu HBase übergibt. Anschließend kann  mithilfe einer \textit{ConnectionFactory} eine Verbindung zur Datenbank hergestellt werden.

\begin{lstlisting}[language=Java]
Configuration config = HBaseConfiguration.create();
config.setInt("timeout", 120000);
config.set("hbase.master", "*10.20.110.61:16006*");
config.set("hbase.zookeeper.quorum","10.20.110.61");
config.set("hbase.zookeeper.property.clientPort", "2186");
Connection connection = ConnectionFactory.createConnection(config);
\end{lstlisting}

Das Aufbauen einer Verbindung zu HBase ist eine sehr aufwändige Operation, weshalb es empfehlenswert ist, diese nur einmal beim Hochfahren des Servers auszuführen.
Beim jedem Datenbank-Zugriff muss diese Verbindung verwendet werden um einen Zugriff auf die jeweilige Tabelle zu gewährleisten.

\begin{lstlisting}[language=Java]
Table table = connection.getTable(TableName.valueOf(tableName));
\end{lstlisting}

Nach dem Abarbeiten der Anfrage muss der Zugriff auf die Tabelle wieder geschlossen werden.
\begin{lstlisting}[language=Java]
table.close();
\end{lstlisting}

Wenn der Server heruntergefahren wird,  sollte auch die Verbindung zur Datenbank geschlossen werden.
Um dies umzusetzen ist es sinnvoll, das Schließen der Verbindung zur Datenbank in eine Methode mit der Annotation @PreDestroy zu implementieren.

\begin{lstlisting}[language=Java]
@PreDestroy
public void closeConnection() throws IOException {
       connection.close();
}
\end{lstlisting}

Die Datenübertragung zwischen dem Client und der Datenbank passiert über \acp{RPC}. Für jeden Aufruf der \textit{next}-Methode auf dem \textit{ResultScanner} wird ein \ac{RPC}  ausgeführt. Bei einer großen Datenmenge könnte diese Vorgehensweise sehr ineffizient sein. Deswegen ist es ratsam bei solchen Aufrufen den Parameter \textit{cache} im \textit{Scanner}-Objekt sinnvoll zu setzen:
\begin{lstlisting}[language=Java]
Scan.setCaching(500)
\end{lstlisting}
Dabei werden bei jedem \ac{RPC} 500 Datensätze zurückgeliefert und im Speicher gehalten.



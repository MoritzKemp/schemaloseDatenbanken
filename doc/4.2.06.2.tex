\subsubsection{Anwendungsszenario}
Dieser Abschnitt beschreibt die Anforderungen an das Softwaresystem mittels Anwendungsfällen und 
beschreibt die technischen Rahmenbedingungen, in dem das Softwaresystem eingebettet ist.
\subsubsection{UseCases}\label{usecases}

Die Anwendungsfälle beschreiben die Arbeit mit dem \textit{One-Million-Song}-Datensatz. 

Das Anwendungsfalldiagramm \ref{anforderungen:usecasediagramm} zeigt alle Anwendungsfälle für
die Arbeit mit dem Million-Song-Datensatz auf dem Hochschul-Cluster.

\begin{figure}[H]
	\centering
	\includegraphics[width=0.5\textwidth]{images/{06.usecases}.pdf}
	\caption{Anwendungsfall-Diagramm, UML 2.0. Alle Anwendungsfälle für das Million-Song-Projekt}
	\label{anforderungen:usecasediagramm}
\end{figure}

Es folgt eine kurze Beschreibung der einzelnen Anwendungsfälle. Auf eine detaillierte Ausarbeitung im Rahmen einer 
ausführlichen Anforderungsanalyse soll in diesem Falle verzichtet werden, weil ein konkretes Anwendungsszenario fehlt und
die Anwendungsfälle künstlicher Natur sind, um den Einsatz von Hadoop, MapReduce und HBase im Bereich einer
Big-Data-Anwendung zu evaluieren.

\begin{description}
	\item[UC 1] Der Nutzer möchte eine Motto-Party veranstalten. Er kann sich allerdings nicht mehr genau an den Namen des Interpreten erinnern. Mit Hilfe der Vorschläge, die ihm dynamisch während des Tippens angezeigt werden, ist es ihm möglich den Interpreten zu finden.
	\item[UC 2] Der Nutzer möchte nun alle Songs zu einem Interpreten aus \textbf{UC 1} anzeigen.
	\item[UC 3] Der Nutzer möchte ein speziell für ihn angepasstes Lauftraining absolvieren. Dazu sucht er nach Songs mit einer speziellen \ac{BPM}-Rate.
	\item[UC 4] Der Nutzer möchte nach besonders erfolgreichen Interpreten in der Datenbank suchen. Daher lässt er sich eine Tabelle ausgeben, in der die Songs pro Interpret dargestellt werden.

\end{description}

Die technischen Rahmenbedingungen des Clusters mit insgesamt 5 Rechnerknoten
 sind allgemein bekannt und werden hier nicht weiter ausgeführt.

\subsubsection{Parametriesierbarkeit}
Sowohl der Anwendungsfall \textit{Finde Interpreten anhand eines Substrings} als auch \textit{Zeige alle Songs zu einem Interpreten an}
beinhalten einen Text-Parameter. Die Definition des Parameters erfolgt in der Anwendung über ein Text-Eingabefeld.

\subsubsection{Installation und Konfiguration}
Auf dem Cluster wird das komplette Hadoop-Softwarepaket installiert, bestehend aus den
Komponenten \ac{YARN}, \ac{HDFS} und MapReduce. Installiert wird über einen Tarball von der offiziellen
Apache-Webseite (\url{http://www-us.apache.org/dist/hadoop/common/hadoop-2.7.3/hadoop-2.7.3.tar.gz}), um die aktuellste Version ($2.7.3$) von Hadoop zu erhalten. Die Installation selbst gestaltet sich mit dem Entpacken der Binär-Dateien im home-Verzeichnis 
denkbar einfach. Komplexer gestaltet sich die Konfiguration, die das Laufzeitverhalten der Komponenten festlegen.

Eine Auflistung der ab Werk ausgelieferten Standard-Konfiguration kann auf folgenden Webseiten nachgeschlagen werden:
\cite{hdfsDefault}, \cite{yarnDefault}, \cite{mapreduceDefault} \cite{hbaseConfig}.

Die im Anhang dieses Dokuments liegenden Tabellen zeigen die Konfiguration von HDFS \ref{config:hdfsValues}, YARN \ref{config:yarnValues}, MapReduce \ref{config:mapreduceValues} und HBase 
speziell für die vorhandene Cluster-Umgebung. Es werden  nur Parameter aufgelistet, 
die von der Werks-Konfiguration abweichen.
HBase liefert von Haus aus bereits eine eigene ZooKeeper-Installation mit. Der Grund für eine eigene, getrennte ZooKeeper-Installation lag in der sauberen Trennung zwischen der HBase-Konfiguration und der ZooKeeper-Konfiguration. 

\subsubsection{Berechnung der Speicher-Konfiguration}
\label{anforderung:berechungSpeicher}
Im Zuge der Probleme mit der Ausführung einer Map-Reduce-Jobs, bei dem die Rechnerknoten
offenbar wegen Überlastung ihren Dienst quittierten, wird die Speicher-Verwendung
des Map-Reduce-Frameworks manuell eingestellt. Der Algorithmus zur Berechnung der 
benötigen Speichermengen ist \cite{memoryCal} Unterpunkt \textit{ 10.2. Manually Calculating YARN and MapReduce Memory Configuration Settings} entnommen.
Grundlage für die Berechnung ist die verfügbare Hardware auf den Rechnerknoten, speziell
die verfügbare Menge an Speicher, die Anzahl an Prozessor-Kernen und die Anzahl an 
Festplatten. Basierend auf der gegebenen Hardware-Spezifikation des Clusters 
ergaben sich die Grund-Parameter in Tabelle
\ref{config:memoryCalculation}.
Dabei wurde der verfügbare Speicher von 32 auf 8 GB gekürzt, um die Speichernutzung der
anderen Teams auf dem Cluster zu berücksichtigen.

\begin{table}
	\begin{tabularx}{\textwidth}{|X|X|} \hline
	Speicher & 8 GB \\ \hline
	Kerne & 8 \\ \hline
	Festplatten & 1 \\ \hline
	\end{tabularx}
	\caption{Grund-Parameter zur Berechnung der Map-Reduce Speicherverwendung}
	\label{config:memoryCalculation}
\end{table}

Als Zwischen-Ergebnis der Berechnung erhält man die maximale Anzahl an parallel laufenden 
Containern pro Rechnerknoten und der die Menge an Speicher pro Container. Für dieses Cluster
ergibt die Berechnung eine Container-Anzahl von $3$ maximal parallelen Containern und 
eine empfohlene Zuweisung von $1536$ Megabyte pro Container. Mit diesen Parametern lässt 
sich nun die empfohlene Speicher-Konfiguration des Map-Reduce-Frameworks berechnen, die
als Konfiguration in den Tabellen \ref{config:yarnValues} und
 \ref{config:mapreduceValues} zu sehen ist.

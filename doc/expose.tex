\chapter{Expose}
\section{Ziel}
Das Ziel der Projektarbeit zum Thema \textit{Hadoop/Hbase} ist es, 
eine Datenbank mittels Hadoop/Hbase zu implementieren, die eine Lieder-Datenbank
darstellt. Dabei soll die Datenbank so installiert werden, dass sie ihre Stärken mit
sehr großen Datensätze, in diesem Falle dem Million-Song-Datensatz, ausspielen kann.

\section{Vorgehensweise}
Das Team wird sich zuerst mit den Grundlagen von NoSQL-Datenbanken befassen, 
vor allem explizit mit den Grundlagen von Hadoop/Hbase. Darauf aufbauend lässt
sich eine Argumentation formulieren, warum die Behandlung von derart großen Datensätzen
wie der Million-Song-Datensatz sich mit Hadoop/Hbase effektiver gestalten lässt als gegenüber
den klassischen relationalen Datenbanken. Dieser Teil beinhaltet also im wesentlichen
Grundlagen zu den Wide-Column-Datenbanken sowie Grundlagen zu Hadoop und Hbase.

Im zweiten Teil soll dann darauf aufbauend die praktische Umsetzung durchgeführt und beschrieben
werden.\textit{Johann, your part! Whoop whooop}

Die geplante Gliederung sieht wie folgt aus:
\begin{itemize}
	\item Grundlagen von Wide-Column-Datenbanken (only MapReduce, 6C)
	\item Grundlagen Hadoop/Hbase (6D1/2)
	\item Anforderungen an die Anwendung
	\item Aufbau der Datenbank (6F)
	\item Installation
	\item Cluster-Betrieb
	\item Benchmarks
\end{itemize}

\section{Meilensteine}

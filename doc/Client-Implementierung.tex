\subsubsection{Oberflächengestaltung}

Während des Semesterprojektes wurden von uns einige UseCases ausgearbeitet, die die Arbeit mit der HBase DB zeigen sollen.\\
Um von uns ausgearbeiteten UseCases sichtbar in einer Oberfläche darzustellen wurde ein Java Client mit einer graphischen Weboberfläche entwickelt.

Der Java Client musste auf demselben Server laufen, wo auch die DB liegt. Da man keinen weiteren Aufwand für das Aufsetzen eines Application Servers betreiben wollte, hat man sich für Spring-Boot entschieden. Der Vorteil von diesem Framework ist, dass beim Starten der Applikation ein lauffähiger Tomcat Applikation Server bereitgestellt wird. 

Aus der Menge der vorhandenen Java Frontend Frameworks wurde Spring MVC gewählt.  

Die UseCases kann man in zwei Arten unterteilen:\\
\begin{itemize}
\item Darstellung der CRUD Operationen
\item Darstellung der Ergebnisse eines MapReduce Prozesses
\end{itemize}
Bei der Darstellung der CRUD Operationen ging es sowohl um den Zugriff auf verschiedene Datensätze innerhalb der DB als auch um das Speichern und Löschen eines Datensatzes.\\
Dabei geht es um die Tabelle \textit{"music"}, die die Datensätze aus der MillionSong DB beinhaltet.


Der Client stellt die Ergebnisse verschiedener Abfragen gegen die HBase Datenbank ab. So z.B. kann man alle Lieder von einem Interpreter in einer Tabelle anzeigen lassen. In der ersten Spalten wurden die RowKeys angezeigt, da man diese für die Aktualisierung eines Datensatzes braucht.

Für die Demonstration der CRUD Operationen hat man im unteren Teil der Tabelle Felder eingeführt in denen man die Daten für einen neuen Datensatz hinzufügen kann. Dieselben Felder werden auch für die Aktualisierung eines Datensatzes verwendet. Dabei muss man den RowKey des Datensatzes eingeben welchen man aktualisieren möchte.
Für das Löschen eines Datensatzes gibt es unter der Tabelle ein Textfeld für die Eingabe des RowKeys und ein "remove" Button. Nach dem Betätigen eines Buttons wird die Aktion ausgeführt und die Tabelle angepasst.

Die weitere Art des UsesCases ist die Darstellung der Ergebnisse, die im Laufe des Map Reduce Jobs ermittelt worden.


\begin{lstlisting}[language=Java]
Configuration config = HBaseConfiguration.create();
            config.setInt("timeout", 120000);
            config.set("hbase.master", "*10.20.110.61:16006*");
            config.set("hbase.zookeeper.quorum","10.20.110.61");
            config.set("hbase.zookeeper.property.clientPort", "2186");
	   Connection connection = ConnectionFactory.createConnection(config);
\end{lstlisting}
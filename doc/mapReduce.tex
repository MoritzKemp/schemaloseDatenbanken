\chapter{Das MapReduce-Framework}
MapReduce wurde 2004 von zwei Mitarbeitern von Google vorgestellt \cite{dean2008mapreduce}. Es ist ein Framework
für die Programmierung von massiver, parallele Datenverarbeitung sehr großen Datenmengen. Der Programmierer
kann sich dabei komplett auf die Verarbeitung der Daten konzentrieren, während das MapReduce-Framework sich um die
Verteilung der Daten und der parallelen Ausführung der Anwendung kümmert. Im Zusammenarbeit mit speziellen Filesystemen 
für große Cluster-Systeme, die zum Beispiel automatisch redundante Kopien auf mehreren Maschinen anlegen, ermöglicht
eine mit MapReduce programmierte Anwendung eine hohe Toleranz gegenüber dem Ausfall von einzelnen Maschinen.

\section{Arbeitsweise von MapReduce}
Auf der obersten Abstraktionsebene besteht das MapReduce-Framework nur aus zwei Funktionen: \textit{map} und \textit{reduce}.
Beide Funktionen müssen vom Anwender des MapReduce-Frameworks spezifiziert werden.
\textit{map} nimmt eine (ungeordnete) Liste von Daten entgegen, die Schlüssel-Wert-Paare (key/value) enthält. Die Funktion 
verarbeitet nun diese Liste nach dem vom Programmierer spezifizierten Funktion und liefert als Ergebnis wiederum eine
Liste von Schlüssel-Wert-Paaren. Die Logik der \textit{map}-Funktion ist vom Anwendungsfall abhängig, bildet aber in der Regel
die ungeordneten Daten in eine Liste ab, die die für den Anwendungsfall interessanten Daten enthält. Ein gern genommenes Beispiel
ist das Zählen von Wörtern in tausenden von Dokumenten. Die Liste für die Eingabe in die \textit{map}-Funktion enthält für jeden Schlüssel
als Wert ein ganzes Dokument. Diese Dokumente werden durchsucht und gleichzeitig eine neue, deutlich größere Liste mit Schlüssel-Wert-Paaren erstellt, 
die für jedes einzelne Wort den String selbst als Schlüssel speichert, und als Wert eine 1 angibt. Somit ist jedes Wort in dieser Liste repräsentiert.
Diese neue Liste enthält sehr viele gleiche Schlüssel, wobei alle den Wert 1 tragen.

Das Ergebnis von der \textit{map}-Funktion ist aber immer nur ein Zwischen-Ergebnis und wird direkt als Eingabe für die \textit{reduce}-Funktion verwendet,
die am Ende das tatsächliche Ergebnis der Verarbeitung ausgibt. Die \textit{reduce}-Funktion hat in der Regel die Aufgabe, das Ergebnis der \textit{map}-Funktion
zusammenzufassen. Bezogen auf das Beispiel mit der Zählung von Wörtern bekommt die \textit{reduce}-Funktion nun eine (sehr lange) Liste von allen Wortvorkommnissen.
Die Funktion sortiert  nun diese Liste und erzeugt eine neue Liste, die alle Schlüssel-Werte enthält, aber nur ein einziges mal in der Liste. Mehrfache Einträge
in der ursprünglichen Liste mit gleichen Schlüsselwerten werden also zu einem Eintrag zusammengefasst, wobei die Werte der einzelne, mehrfachen Einträge
aufaddiert werden. Das Ergebnis ist eine Liste mit jedem Wort, das in den Dokumenten vorkommt, als einmaliger Schlüssel und als Wert die jeweilige Häufigkeit.


\section{Anwendung für Datenbanken}

\section{Optimierung}

\section{Beispiel}
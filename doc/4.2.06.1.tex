\subsubsection{Import der Daten}
Um die Verwendung von der NoSQL-Datenbank HBase und das Programmiermodel MapReduce zu demonstrieren, wird das Million-Song-Dataset herangezogen. Die Datensätze werden von dem Anbieter \url{http://labrosa.ee.columbia.edu/millionsong/pages/getting-dataset} bereitgestellt und  sind im Verzeichnis \textit{/data/team6/MillionSongSubset/data} in den \textit{*.h5}-Dateien zu finden. 

Die erste Aufgabe liegt darin die Daten aus dem \textit{*.h5}-Format in die HBase-Datenbank zu importieren.
Dafür wird die Java Bibliothek \textit{ncsa.hdf.object.h5} verwendet. Mit Hilfe dieser Bibliothek kann auf die Daten innerhalb der \textit{*.h5}-Datei zugegriffen werden.

Für die Usecase-relevanten Daten wird in HBase die Tabelle \textbf{music} angelegt. Bei der Analyse der Struktur für die Tabelle fällt auf, dass für die anvisierten Use Cases nicht alle Spalten gebraucht werden. Deswegen wird ein Datenschema wie in Abbildung \ref{fig:datenschema} verwendet. Die irrelevanten Daten für die Spaltenfamilie \textbf{miscellaneous} werden aus der \textit{*.h5}- Datei ausgelesen und als ein konkatinierter String abgespeichert. Dabei sind die einzelnen Daten durch ein Semikolon voneinander getrennt. An dieser Stelle ist es nun die Aufgabe des Softwareentwicklers die Daten bei der Implementierung der Applikation auseinander zu parsen.

Nachdem die Tabelle mit den entsprechenden Spalten erstellt ist, können die Daten aus den \textit{*.h5}- Dateien mit Hilfe eines Java Programms direkt in HBase importiert werden.

Im Folgenden sind ein paar Codefragmente aus dem Datenimport dargestellt.
Um die Daten aus einer \textit{*.h5}-Datei auslzulesen, wird eine Verbindung zu einer \textit{*.h5}-Datei erstellt:


\lstset{
    language=Java,
    basicstyle=\ttfamily,
    frame=single,
    breaklines=true,
    postbreak=\raisebox{0ex}[0ex][0ex]{\ensuremath{\hookrightarrow\space}}
}

\begin{lstlisting}[language=Java]%[caption={fgdfgfd}, label=mapreduce:dgdgs]

H5File h5File = new H5File(filename, H5File.READ)
\end{lstlisting}
Mit dem Wissen über die Struktur der Daten innerhalb der \textit{*.h5}-Datei kann man auf die einzelnen Werte zugreifen. Dies wird exemplarisch für den Zugriff auf den Wert \textit{analysis\_sample\_rate} in der Tabelle \textit{/analysis/songs} gezeigt:\\

\begin{lstlisting}[language=Java]
public int getSampleRate(H5File h5File){
        H5CompoundDS analysis = (H5CompoundDS) h5File.get("/analysis/songs");
        analysis.init();
        int wantedMember = find( analysis.getMemberNames() , "analysis_sample_rate");
        assert(wantedMember >= 0);
        Vector alldata = (Vector) analysis.getData();
        int[] col = (int[]) alldata.get(wantedMember);
        return col[songidx];
}
\end{lstlisting}

Im Folgenden wird die Vorgehensweise für das Schreiben in die HBase-Datenbank gezeigt. Wie die Verbindung zu HBase erstellt wird und was dabei zu beachten ist, wird in Abschnitt \ref{schnittstelle} beschrieben.

Nachdem man den Zugriff zur Tabelle hergestellt hat, kann man verschiedene Operationen ausführen.
Um einen neuen Datensatz hinzuzufügen, erzeugt man eine Datenreihe mit dem Put-Objekt:

\begin{lstlisting}[language=Java]
Put p = new Put(Bytes.toBytes("Song1")); /* Erzeugen ein Datensatz mit dem RowKey = "Song1'' */
p.addColumn(Bytes.toBytes("song"), Bytes.toBytes("Title"),Bytes.toBytes("HISTORY")); /* Erzeuge fuer diesen RowKey inder Spaltenfamileie "Song" die Splate "Title" mit dem Wert "HISTORY" */
table.put(p);
\end{lstlisting}

Defaultmäßig wird jede PUT-Operation sofort an die Datenbank geschickt. Wenn man eine große Menge an Daten hat, die in die Datenbank eingetragen werden muss, dann ist es aus den Performanz-Gründen sehr ratsam AutoFlash auszuschalten. \\
Das erreicht man wie im folgt:
\begin{lstlisting}[language=Java]
htable.setAutoFlush(true);
\end{lstlisting}
Hierbei werden die Daten nur dann an die Datenbank übertragen, wenn der \textit{Write-Buffer} voll ist. Die Größe eines \textit{Write-Buffers} liegt standardmäßig bei 2MB. 




\begin{abstract}
\section*{Zusammenfassung}\markboth{Zusammenfassung}{}
  \addcontentsline{toc}{chapter}{Zusammenfassung}
  Das Ziel der Projektarbeit zum Thema \textit{Hadoop/Hbase} ist es, 
eine Datenbank mittels Hadoop/Hbase zu implementieren, die eine Lieder-Datenbank
darstellt. Dabei soll die Datenbank so installiert werden, dass sie ihre Stärken mit
sehr großen Datensätzen, in diesem Falle dem Million-Song-Datensatz, ausspielen kann.

Das Team wird sich zuerst mit den Grundlagen von NoSQL-Datenbanken befassen, 
vor allem explizit mit den Grundlagen von Hadoop/Hbase. Darauf aufbauend lässt
sich eine Argumentation formulieren, warum sich die Behandlung von derart großen Datensätzen
wie dem Million-Song-Datensatz mit Hadoop/Hbase effektiver gestalten lässt als mit klassischen relationalen Datenbanken. Dieser Teil beinhaltet also im wesentlichen
Grundlagen zu den Wide-Column-Datenbanken sowie Grundlagen zu Hadoop und Hbase.

Im praktischen Teil wird zunächst die benötigte Software installiert, i.e. Java, Hadoop und HBase. Um die Installation auf einer Windows-Maschine auszuführen wird zusätzlich Cygwin installiert. Um auf die HBase-Datenbank zugreifen zu können, muss HBase zunächst konfiguriert werden.

Nach erfolgreicher Installation wird HBASE zunächst im Standalone-Modus auf einer Maschine betrieben. Zur Anschauung wird eine Tabelle angelegt und es werden CRUD-Operationen geschildert. Im Anschluss daran wird gezeigt, wie man programmatisch mit der Datenbank arbeiten kann mit Hilfe von JRuby.

Nach den ersten Tests wird der Import von Daten (One-Million-Song-Datensatz) beschrieben und durchgeführt. Für die Verwaltung der Daten wird eine Client-Anwendung mit Ruby geschrieben.

Im letzten Schritt wird HBase auf dem Hochschul-Cluster installiert und für den Cluster-Modus konfiguriert.

Die geplante Gliederung sieht wie folgt aus:
\begin{itemize}
	\item Grundlagen von Wide-Column-Datenbanken (only MapReduce, 6C)
	\item Grundlagen Hadoop/Hbase (6D1/2)
	\item Anforderungen an die Anwendung
	\item Aufbau der Datenbank (6F)
	\item Installation
	\item Cluster-Betrieb
	\item Benchmarks
\end{itemize}

Folgende Meilensteine werden definiert:
\begin{itemize}
\item[28.10] - Expose
\item[25.11] - HBase im Standalone-Modus auf einer Windows-Maschine installieren und CRUD-Operationen mit Testdaten durchführen.
\item[25.11] - One-Million-Dataset importieren
\item[02.12] - Zwischenbericht
\item[02.12] - MapReduce-Kapitel 6C
\item[02.12] - Hadoop-Kapitel 6D1  
\item[16.12] - HBase-Kapitel 6D2 
\item[16.12] - Client-Anwendung fertig stellen
\item[23.12] - Cluster-Installation abgeschlossen
\item[06.01] - Ausarbeitung Projektarbeit
\end{itemize}

\end{abstract}

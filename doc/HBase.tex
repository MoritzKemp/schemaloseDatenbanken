\section{HBase}

\cite{Redt01} embeddeed in text.
\cite{SpaOd16} embeddeed in text.


Im folgenden wird HBase als Datenbank vorgestellt und anschließend werden Gründe aufgezeigt, warum HBase im Rahmen von BigData eingesetzt werden sollte.

\subsection{Entstehung}
Nachdem Google immer größer Datenmassen speichern musste und das mit dem GFS gelöst schien, stellte sich ein weiteres Problem heraus: Die Indexierung dieser Daten, die nun verteilt auf mehreren Knoten eine hohe Konsistenz und schnelle Schreibe- und Lesezugriffe gewährleisten sollte. Auf Grundlage eines veröffentlichten Whitepapers zu BigTable entwickelte die Opensource-Community Hbase, weshalb diese Datenbanken Gemeinsamkeiten bezüglich ihrer Funktionalität aufweisen. Beispielsweise unterstützen beide die Komprimierung und Versionierung der Daten.

\subsection{Allgemein}
HBase bietet im Allgemeinen einen schnellen (nahe Echtzeit) Zugriff auf riesige Datenmengen. HBase ist eine Datenbank, die diese Datenmengen, auf mehreren Knoten verteilt, verwaltet und jederzeit erweitert werden kann. Sie basiert auf Java, ist Open source, nicht-relational, spaltenrientiert und setzt auf ein verteiltes Dateisystem wie HDFS von Hadoop auf. Es wurde als fehlertolererantes System entworfen, das auch unvollständige Datenmengen zu speichern weiß. Nach CAP-Theorem legt es bsonders Wert auf Verfügbarkeit und Partitionierung und vernachlässigt die Konsistenz der Daten. Des weiteren unterstützt HBase Replikation, den MapReduce-Algorithmus, automatische Verteilung der Tabellen auf die Knoten, automatische Verteilung der Last, Kompremierung der Daten und Bloom-Filter. 

\subsection{Portfolio}

\subsection{Abgrenzung}
\subsubsection{Hbase im Vergleich zu einem relationalen Datenbank-Managementsystem}
HBase speichert die teils unvollständigen Daten in Spalten ab im Vergleich zu vollständigen Reiheneinträgen in einem RDBMS. Dies ist notwendig da große Datenmengen den Anforderungen auf Vollständigkeit, wie sie ein RDBMS erfordert, oftmals nicht gerecht werden.  
\subsubsection{Hbase im Vergleich zu HDFS}



\subsection{Datenmodell}

\subsection{Tabellenformat}
\subsection{Datenspeicherung}
\subsection{Interne Tabellenoperationen}
\subsubsection{Komprimierung}
\subsubsection{Automatische Verteilung auf die Knoten}
\subsubsection{Verteilung der Last}

\subsection{Master-Slave-Sytem}

Hbase benütigt kein Schema und unterstützt ein flexibles Datenmodell, indem ein Hinzufügen einer neuen Spalte jederzeit mögich ist. Daten werden in Tabellen gespeichert.

\subsection{Zugriff auf HBase}
\begin{itemize}

\item JAVA
\item REST
\item Avro
\item Thrift
\end{itemize}



Des weiteren erinnert HBase an eine relationale Datenbank, da sie ihre Daten in Tabellen speichert, die Zellen enthalten. Jedoch verhalten sich die Tabellen nicht wie Relationen und Zeilen nicht wie Datensätze in relationalen Datenbanken. Auch sind die Spalten nicht durch ein Schema definiert.

Da HDFS ein Filesystem ist, fehlt ihm die zufällige Lese-und Schreibfähigkeit. Eine mögliche Lösung dafür ist HBase. Es wird innerhalb des Hadoop-Clusters betrieben und stellt in Echtzeit Lese- und Schreibzugriff zu den Daten her. HBase ist besonders wertvoll, wenn es um die Verarbeitung von sehr großen Datenmengen geht. Aus diesem Grund wird es auch oftmals als Logging- und Suchsystem in großen Unternehmen eingesetzt.

\subsection{Gründe für HBase}
Für den Einsatz von HBase gibt es gleich mehrere Gründe:
\begin{itemize}

\item Skalierbarkeit
\item Versionierung
\item Komprimierung
\item Garbage Collection
\item speicherbasierte Tabellen
\item Durch write-ahead-Logging und eine verteilte Konfiguration kann sich Hbase schnell von Serverausfällen erholen
\item Bietet geringe Latenz bei Zugriff auf kleine Teil-Datenmengen 
\item Bietet ein flexibles Datenmodel
\end{itemize}
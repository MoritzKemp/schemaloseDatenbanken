\begin{abstract}
\section*{Zusammenfassung}\markboth{Zusammenfassung}{}
  \addcontentsline{toc}{chapter}{Zusammenfassung}
  Das Ziel der Projektarbeit zum Thema \textit{Hadoop/Hbase} ist es, 
eine Datenbank mittels Hadoop/Hbase auf einem bereitgestelltem
Cluster so zu installieren, dass der Anwender in dem One-Million-Datensatz nach Informationen zu Musikstücken
suchen kann. 

%Im ersten Teil der Projektarbeit werden die theoretischen Grundlagen von Hadoop/Hbase 
%behandelt. Dieser Teil beinhaltet ein Kapitel zu Map/Reduce, ein Kapitel zu Hadoop und ein Kapitel zu Hbase.
%Die Theorie soll etwa ein Drittel der Projektarbeit ausmachen. 

Im ersten Teil der Projektarbeit werden die technologischen Grundlagen zu MapReduce
behandelt.Die Theorie soll etwa ein Drittel der Projektarbeit ausmachen. Im zweiten Kapitel werden das NoSQL-System HBase und Hadoop vorgestellt.
Hier wird auch Schritt für Schritt das Vorgehen  beschrieben, wie die Installation und Konfiguration 
der Systeme abläuft. Fragestellungen werden sein, wie 
sich die Datenbank auf einen Cluster installieren lässt und welche Konfigurationsparameter hinsichtlich des
Anwendungsfalls eines One-Million-Datensatzes berücksichtigt und angepasst werden müssen.

Der dritte Teil der Projektarbeit beschreibt die praktische Umsetzung, die aus der Implementierung der Client-Software besteht.
Eingangs werden Anforderungen, Anwendungsfälle und technische Rahmenbedingungen kurz beschrieben,
um einen Überblick zu geben.   %Eingesetzte Programmiersprache ist hier Java.
Auch die konkrete
Implementierung des MapReduce-Algorithmus wird in diesem Kapitel dargelegt. Des Weiteren werden Filter- und Optimierungsparameter,
die besonders HBase betreffen, evaluiert und gegebenenfalls angewendet.

Außerdem wird gezeigt,
ob und wie sich die Datenbank auch ohne eigens programmiertem Client Ad-Hoc ansprechen lässt, beispielsweise über
eine laufende Shell. Das nächste Kapitel beschreibt die Implementierung der Client-Software, die mittels Java als Serveranwendung
und Java als Client-Anwendung dem Anwender die Möglichkeit bieten soll, über eine grafische Oberfläche mit dem One-Million-Datensatz zu arbeiten. Der Client beschränkt sich dabei auf wenige, nur funktionale Anforderungen. Eine aufwendig grafische Oberfläche ist nicht das Ziel.
Ein weiteres Kapitel im praktischen Teil behandelt die Installation des One-Million-Song-Datensatzes in das Hadoop-Dateisystem.
Wichtige Fragestellungen sind dabei, inwieweit die Daten bereits für die Speicherung im Hadoop-Dateisystem geeignet sind und
die Frage nach der Replikation auf dem Cluster-System.

Zum Abschluss des praktischen Teils folgt eine kurze Leistungsdarstellung des Systems, indem die Antwortzeiten
der Datenbank zu bestimmten Abfragen gemessen werde. Optional ist eine Leistungsbewertung im Vergleich mit 
einem anderen Team denkbar, die ebenfalls mit dem selben Datensatz arbeiten.


%Das Team wird sich zuerst mit den Grundlagen von NoSQL-Datenbanken befassen, 
%vor allem explizit mit den Grundlagen von Hadoop/Hbase. Darauf aufbauend lässt
%sich eine Argumentation formulieren, warum sich die Behandlung von derart großen Datensätzen
%wie dem Million-Song-Datensatz mit Hadoop/Hbase effektiver gestalten lässt als mit klassischen relationalen Datenbanken. Dieser Teil beinhaltet also im wesentlichen
%Grundlagen zu den Wide-Column-Datenbanken sowie Grundlagen zu Hadoop und Hbase.

%Im praktischen Teil wird zunächst die benötigte Software installiert, i.e. Java, Hadoop und HBase. Um die Installation auf einer Windows-Maschine auszuführen wird zusätzlich Cygwin installiert. Um auf die HBase-Datenbank zugreifen zu können, muss HBase zunächst konfiguriert werden.

%Nach erfolgreicher Installation wird HBASE zunächst im Standalone-Modus auf einer Maschine betrieben. Zur Anschauung wird eine Tabelle angelegt und es werden CRUD-Operationen geschildert. Im Anschluss daran wird gezeigt, wie man programmatisch mit der Datenbank arbeiten kann mit Hilfe von JRuby.

%Nach den ersten Tests wird der Import von Daten (One-Million-Song-Datensatz) beschrieben und durchgeführt. Für die Verwaltung der Daten wird eine Client-Anwendung mit Ruby geschrieben.

%Im letzten Schritt wird HBase auf dem Hochschul-Cluster installiert und für den Cluster-Modus konfiguriert.

%\begin{itemize}
%\item[28.10] - Expose
%\item[25.11] - HBase im Standalone-Modus auf einer Windows-Maschine installieren und CRUD-Operationen mit Testdaten durchführen.
%\item[25.11] - One-Million-Dataset importieren
%\item[02.12] - Zwischenbericht
%\item[02.12] - MapReduce-Kapitel 6C
%\item[02.12] - Hadoop-Kapitel 6D1  
%\item[16.12] - HBase-Kapitel 6D2 
%\item[16.12] - Client-Anwendung fertig stellen
%\item[23.12] - Cluster-Installation abgeschlossen
%\item[06.01] - Ausarbeitung Projektarbeit
%\end{itemize}

\end{abstract}

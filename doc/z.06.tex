\begin{abstract}
\section*{Zusammenfassung}\markboth{Zusammenfassung}{}
  \addcontentsline{toc}{chapter}{Zusammenfassung}
  Das Ziel der Projektarbeit zum Thema \textit{Hadoop/Hbase} ist es, 
eine Datenbank mittels Hadoop/Hbase auf einem bereitgestelltem
Cluster so zu installieren, dass der Anwender in dem One-Million-Datensatz nach Informationen zu Musikstücken
suchen kann. 


Im ersten Kapitel werden die technologischen Grundlagen zu MapReduce behandelt. Im dritten Kapitel werden das Framework Hadoop und das  NoSQL-System HBase untersucht.
Am Ende der jeweiligen Unter-Kapitel wird auch beschrieben, wie die Installation und die Konfiguration 
der Systeme abläuft. Fragestellungen werden sein, wie 
sich die Datenbank auf einen Cluster mit fünf Knoten installieren lässt und welche Konfigurationsparameter hinsichtlich des
Anwendungsfalls eines One-Million-Datensatzes berücksichtigt und angepasst werden müssen. Außerdem wird gezeigt,
ob und wie sich die Datenbank auch ohne eigens programmiertem Client Ad-Hoc ansprechen lässt, beispielsweise über
eine laufende Shell. 

Das vierte Kapitel der Projektarbeit beschreibt die praktische Umsetzung, die aus der Implementierung der Client-Software in JAVA und des Dateminports besteht.
Wichtige Fragestellungen sind dabei, inwieweit die Daten bereits für die Speicherung im Hadoop-Dateisystem geeignet sind wie die Daten auf dem Cluster-System repliziert werden.
Ebenfalls werden in diesem Kapitel die Use cases vorgestellt, die mit der Client-Software umgesetzt werden. Der Client beschränkt sich dabei auf wenige, nur funktionale Anforderungen. Eine aufwendig grafische Oberfläche ist nicht das Ziel.
Auch die konkrete Implementierung von MapReduce-Jobs wird in diesem Kapitel erläutert. 

Das letzte Kapitel beschreibt die Erfahrungen, die während des Projekts gemacht wurden, zeigt die Vor- und Nachteile auf und gibt einen Ausblick auf die Zukunft der verwendeten Technologie.

\end{abstract}

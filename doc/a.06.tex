\begin{appendices}
\chapter{Anhang}

\section{Map-Reduce Quellcode}
\lstinputlisting[language=Java, caption={Map-Reduce für das Enfernen von mehrfachen Songeinträgen},label=mrCompleteToStripped]{sourcecode/CompleteToStripped.java}

\lstinputlisting[language=Java, caption={Map-Reduce für das Zählen von Songs pro Künstler}, label=mrCountArtistSongs]{sourcecode/CountArtistSongs.java}

\lstinputlisting[language=Java, caption={Map-Reduce für das Zählen von Songs eines bestimmten Themas}, label=mrCountSongsWithTopic]{sourcecode/CountSongsWithTopic.java}

\section{Unit-Tests}

\lstinputlisting[language=Java, caption={Unit-Test für \ref{mrCompleteToStripped}},label=testCompleteToStripped]{sourcecode/CompleteToStrippedTest.java}

\lstinputlisting[language=Java, caption={Unit-Test für \ref{mrCountArtistSongs}}, label=testCountArtistSongs]{sourcecode/CountArtistSongsTest.java}

\lstinputlisting[language=Java, caption={Unit-Test für \ref{mrCountSongsWithTopic}}, label=testCountSongsWithTopic]{sourcecode/CountSongsWithTopicTest.java}

\newpage
\section{Konfiguration}

\begin{table}
	\begin{tabularx}{\textwidth}{| X | X |} \hline
		Name des Parameters & Gesetzter Wert \\ \hline
		dfs.namenode.name.dir & file:/data/team6/namenode \\ \hline
		dfs.namenode.http-address & 10.20.110.61:50071  \\ \hline
		dfs.datanode.data.dir & file:/data/team6/datanode \\ \hline
		dfs.datanode.address & hdfs://localhost:5001 \\ \hline
		dfs.datanode.http-address & localhost:50016 \\ \hline
		dfs.datanode.ipc.address & localhost:50026 \\ \hline
	\end{tabularx}
	\caption{Gesetzte Parameter-Werte für die Konfiguration des HDFS}
	\label{config:hdfsValues}
\end{table}

\begin{table}
	\begin{tabularx}{\textwidth}{|X|X|} \hline
		Name des Parameters & Beschreibung \\ \hline
		dfs.namenode.name.dir & Lokaler Dateisystempfad, wo
		HDFS den Log für die Transaktionen ablegen soll. \\ \hline
		dfs.namenode.http-address & Die Web-Adresse, über den die
		Weboberfläche des HDFS-Überwachungswerkzeug erreichbar ist. \\ \hline
		dfs.datanode.data.dir & Lokaler Dateisystempfad, wo HDFS
		die Daten des virtuelle Dateisystems ablegen soll. \\ \hline
		dfs.datanode.address & Die Netzwerk-Schnittstelle, über den 
		der Datanode des Rechnerknotens über das hdfs-Protokoll erreichbar ist. Darüber 
		werden die Daten zwischen den Knoten des HDFS-Dateisystems ausgetauscht. \\ \hline
		dfs.datanode.http-address & Die Web-Adresse der Datanode-
		Adminoberfläche. \\ \hline
		dfs.datanode.ipc.address & Der Zugang zum Datanode mittels dem
		leichtgewichtigem IPC-Protokoll, über den Meta-Informationen des Dateisystems 
		ausgetauscht werden. \\ \hline
	\end{tabularx}
	\caption{Beschreibung der Konfigurations-Parameter des HDFS}
	\label{config:hdfsDescription}
\end{table}

\begin{table}
	\begin{tabularx}{\textwidth}{| X | X |} \hline
	yarn.resourcemanager.hostname & 10.20.110.61 \\ \hline
	yarn.resourcemanager.address & \$\{yarn.resourcemanager.hostname\}:8036 \\ \hline
	yarn.resourcemanager.scheduler & \$\{yarn.resourcemanager.hostname\}:8032 \\ \hline
	yarn.resourcemanager.webapp.address & \$\{yarn.resourcemanager.hostname\}:8086 \\ \hline
	yarn.resourcemanager.admin.address & \$\{yarn.resourcemanager.hostname\}:8037 \\ \hline
	yarn.nodemanager.address & \$\{yarn.nodemanager.hostname\}:8056 \\ \hline
	yarn.nodemanager.localizer.address & \$\{yarn.nodemanager.hostname\}:8046 \\ \hline
	yarn.nodemanager.resource.memory-mb & 4608 \\ \hline
	yarn.scheduler.minimum-allocation-mb & 1536 \\ \hline
	yarn.scheduler.maximum-allocation-mb & 4608 \\ \hline
	yarn.nodemanager.resource.percentage-physical-cpu-limit & 50 \\ \hline
	\end{tabularx}
	\caption{Gesetzte Parameter-Werte für die Konfiguration von YARN}
	\label{config:yarnValues}
\end{table}

\begin{table}
	\begin{tabularx}{\textwidth}{| X | X |} \hline
	yarn.resourcemanager.hostname & Der Standort des Resourcemanager von YARN \\ \hline
	yarn.resourcemanager.address &  Der Netzwerk-Port des Resourcemanagers, über den YARN-Jobs gestartet werden können.\\ \hline
	yarn.resourcemanager.scheduler & Die Netzwerk-Schnittstelle des Schedulers\\ \hline
	yarn.resourcemanager.webapp.address & Die Adresse der YARN-Weboberfläche\\ \hline
	yarn.resourcemanager.admin.address & Die Netzwerk-Schnittstelle zum Administrieren des YARN-Resourcemanagers\\ \hline
	yarn.nodemanager.address & Die Netzwerk-Schnittstelle des Nodemanagers von YARN auf den Rechnerknoten\\ \hline
	yarn.nodemanager.localizer.address & Die Netzwerk-Schnittstelle, über die mittels IPC
	Informationen über verfügbare Resourcen von den Rechnerknoten gesammelt wird.\\ \hline
	yarn.nodemanager.resource.memory-mb &  Menge an Speicher, die für einen Container reserviert werden kann.\\ \hline
	yarn.scheduler.minimum-allocation-mb &  Die minimale Menge an Speicher, die vom Resourcemanager für Container angefragt wird\\ \hline
	yarn.scheduler.maximum-allocation-mb &  Die maximale Menge an Speicher, die vom Resourcemanager für Container angefragt wird\\ \hline
	yarn.nodemanager.resource.percentage-physical-cpu-limit & Die maximale Beanspruchung der Rechenkapazität in Prozent insgesamt für alle Container auf einem Knoten. \\ \hline
	\end{tabularx}
	\caption{Beschreibung der Konfigurations-Parameter von YARN}
	\label{config:yarnDescription}
\end{table}

\begin{table}
	\begin{tabularx}{\textwidth}{| X | X |} \hline
	mapreduce.jobhistory.address & 0.0.0.0:10026 \\ \hline
	mapreduce.jobhistory.webapp.address & 0.0.0.0:19886 \\ \hline
	mapreduce.jobhistory.admin.address & 0.0.0.0:10036 \\ \hline
	mapreduce.cluster.local.dir & /data/team6/intermediate \\ \hline
	mapreduce.framework.name & yarn \\ \hline
	mapreduce.map.memory.mb & 1536 \\ \hline
	mapreduce.map.java.opts & -Xmx1228m \\ \hline
	mapreduce.reduce.memory.mb & 3072 \\ \hline
	mapreduce.reduce.java.opts & -Xmx2457m \\ \hline
	yarn.app.mapreduce.am.resource.mb & 3072 \\ \hline
	yarn.app.mapreduce.am.command-opts & -Xmx2457m \\ \hline
	\end{tabularx}
	\caption{Gesetzte Parameter-Werte für die Konfiguration von MapReduce}
	\label{config:mapreduceValues}
\end{table}

\begin{table}
	\begin{tabularx}{\textwidth}{| X | X |} \hline
	mapreduce.jobhistory.address & Die Netzwerk-Schnittstelle mittels dessen Informationen
	über laufende Jobs auf dem Rechnerknoten abgerufen werden können. \\ \hline
	mapreduce.jobhistory.webapp.address & Die Web-Oberfläche für die Betrachtung
	der laufenden MapReduce-Jobs auf dem Rechnerknoten.\\ \hline
	mapreduce.jobhistory.admin.address & Die Netzwerk-Schnittstelle für den Admin für
	Informationen über MapReduce-Jobs.\\ \hline
	mapreduce.cluster.local.dir &  Lokaler Dateipfad, wo die Zwischen-Ergebnisse des MapReduce-Jobs gespeichert werden sollen.\\ \hline
	mapreduce.framework.name &  Spezifiziert, welches konkretes MapReduce-Framework
	verwendet werden soll. \\ \hline
	mapreduce.map.memory.mb &  Verfügbarer Speicher für Map-Jobs\\ \hline
	mapreduce.map.java.opts & Spezifische Java-Optionen. Hier in diesem Falle die Erweiterung
	des verfügbaren Heap-Speichers für die Java-Prozesse von Map-Jobs.\\ \hline
	mapreduce.reduce.memory.mb &  Verfügbarer Speicher für Reduce-Jobs.\\ \hline
	mapreduce.reduce.java.opts & Spezifische Java-Optionen. Hier in diesem Falle die Erweiterung
	des verfügbaren Heap-Speichers für die Java-Prozesse von Reduce-Jobs. \\ \hline
	yarn.app.mapreduce.am.resource.mb & Die Menge an Speicher, die der AppMaster von
	Map-Reduce verwenden kann. \\ \hline
	yarn.app.mapreduce.am.command-opts &  Spezifische Java-Optionen für den AppMaster
	von Map-Reduce, in dem Falle die Erweiterung des Heap-Speichers.\\ \hline
	\end{tabularx}
	\caption{Beschreibung der Konfigurations-Parameter von MapReduce}
	\label{config:mapreduceDescription}
\end{table}

\begin{table}
	\begin{tabularx}{\textwidth}{| X | X |} \hline
	hbase.rootdir & hdfs://10.20.110.61:8026/hbase \\ \hline
	hbase.master.hostname & 10.20.110.61 \\ \hline
	hbase.master.port & 16006 \\ \hline
	hbase.cluster.distributed & true \\ \hline
	hbase.zookeeper.property.clientPort & 2186 \\ \hline
	hbase.zookeeper.quorum & $10.20.110.61$, $10.20.110.43$, $10.20.110.41$, $10.20.110.39$, $10.20.110.48$ \\ \hline
	\end{tabularx}
	\caption{Gesetzte Parameter-Werte für die Konfiguration von Hbase}
	\label{config:hbaseValues}
\end{table}

\begin{table}
	\begin{tabularx}{\textwidth}{| X | X |} \hline
	hbase.rootdir &  Hbase benötigt die Adresse eines Filesystems, wo es die Tabellen ablegen kann. In dem Projekt ist es das installierte
	HDFS-Dateisystem, dessen Zugang über den Namenode des Masters erfolgt.\\ \hline
	hbase.master.hostname & Angabe, auf welchem Rechner-Knoten der Masternode liegen soll.\\ \hline
	hbase.master.port & Der Port, auf dem der Masternode für die Datanodes von Hbase ansprechbar ist. \\ \hline
	hbase.cluster.distributed & Hbase kann entweder als Stand-Alone-Anwendung auf einem Knoten laufen (\textit{false}) oder als verteilte
	 Anwendung (\textit{true}). \\ \hline
	hbase.zookeeper.property.clientPort & Hbase benötigt im verteilten Modus zwingend eine ZooKeeper-Instanz. Hier wird der Port vom
	 installiertem ZooKeeper-Server übergeben.\\ \hline
	hbase.zookeeper.quorum &  Definiert, über welche Rechner-Knoten Hbase verteilt arbeiten soll. \\ \hline
	\end{tabularx}
	\caption{Beschreibung der Konfigurations-Parameter von Hbase}
	\label{config:hbaseDescription}
\end{table}
\newpage

\section{Leistunsgtest}

\lstinputlisting[language=Bash, caption={Ausgabe des Map-Reduce-Jobs für den Leistungstest}, label=performanceLog]{sourcecode/performanceJob.txt}

\end{appendices}

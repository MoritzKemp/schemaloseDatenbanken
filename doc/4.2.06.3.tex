\subsubsection{Oberflächengestaltung}

Während des Semesterprojektes wurden UseCases definiert, die die Arbeit mit  HBase demonstrieren sollen (siehe Abschnitt \ref{usecases}).
Um die UseCases sichtbar in einer Oberfläche darstellen zu können, wurde ein Java-Client mit einer graphischen Weboberfläche entwickelt.

Der Java-Client muss auf demselben Server laufen, wo auch der \textit{MasterServer} läuft. Da kein weiterer Aufwand für das Aufsetzen eines \textit{Application-Servers} betrieben werden soll, wird Spring-Boot eingesetzt. Der Vorteil von diesem Framework ist, dass beim Starten der Applikation ein lauffähiger \textit{Tomcat-Application-Server} bereitgestellt wird. 
Aus der Menge der vorhandenen Java-Frontend-Frameworks wurde \textit{Spring-MVC} gewählt.  

Die UseCases kann man in zwei Gruppen unterteilen:
\begin{itemize}
\item Darstellung der \ac{CRUD}-Operationen.
\item Darstellung der Ergebnisse eines MapReduce-Jobs, beispielsweise die Berechnung der Anzahl der Lieder pro Interpret.
\end{itemize}
Bei der Darstellung der \ac{CRUD}-Operationen geht es sowohl um den Zugriff auf verschiedene Datensätze innerhalb von HBase, als auch um das Speichern und Löschen eines Datensatzes.

Betrachtet wird die Tabelle \textbf{music}, die die Datensätze aus dem MillionSong-Dataset beinhaltet.
Der Client stellt die Ergebnisse verschiedener Abfragen gegen die HBase-Datenbank vor. So ist es beispielsweise möglich, alle Lieder von einem Interpreter in einer Tabelle anzuzeigen. In der ersten Spalte werden die Reihenschlüssel angezeigt, da man diese für die Aktualisierung eines Datensatzes braucht.

Für die Demonstration der \ac{CRUD}-Operationen sind im unteren Teil der Tabelle Felder eingeführt, in denen man die Daten für einen neuen Datensatz hinzufügen kann. Dieselben Felder werden auch für die Aktualisierung eines Datensatzes verwendet. Dabei muss man den Reihenschlüssel des Datensatzes eingeben, den man aktualisieren möchte.
Für das Löschen eines Datensatzes gibt es unter der Tabelle ein Textfeld. Hier kann der Reihenschlüssel angegeben werden und der Datensatz mit Betätigen des \textit{Remove}-Buttons gelöscht werden. Im Anschluss wird die Tabelle aktualisiert.

Die zweite Gruppe der UseCases ist die Darstellung der Ergebnisse der MapReduce-Jobs.
Der MapReduce Job analysiert die Tabelle "music" und stellt zu jedem Interpreter eine Statistic zusammen, wie viele Lieder ihm zugeordnet sind. Diese Daten werden dann in die neue Tabelle "statictics" zusammen. Um das Ergebnis des Jobs zu sehen, kann man in der Client Oberfläche den Button "Anzeige der Analyse" klicken. Dabei werden alle Datensätze dargestellt.

Bei der Implementierung des Clients werden alle Daten auf einer Seite dargestellt. Es wurde weder  Pagination noch srollable Table implementiert, da es für die Darstellung der Arbeit mit der HBase DB keinen Mehrwert bietet.



\subsubsection{Schnittstelle der Anwendung zur Datenbank}

Damit unsere Client Anwendung die Daten aus der HBase DB bekommen kann, muss man eine Verbindung zur Datenbank aufbauen.
Das erreicht man indem man ein Configurationobjekt  erstellt und ihm alle benötigten Daten zu der HBase DB übergibt. Anschlißend kann man mit Hilfe einer ConnectionFactory eine Verbindung zur Datenbank herstellen.

\begin{lstlisting}[language=Java]
Configuration config = HBaseConfiguration.create();
config.setInt("timeout", 120000);
config.set("hbase.master", "*10.20.110.61:16006*");
config.set("hbase.zookeeper.quorum","10.20.110.61");
config.set("hbase.zookeeper.property.clientPort", "2186");
Connection connection = ConnectionFactory.createConnection(config);
\end{lstlisting}

Das Aufbauen einer Verbindung zur DB ist eine sehr Aufwändige Operation und deswegen ist es empfehlenswert diese nur ein Mal beim Hochfahren des Servers auszuführen.
Beim jeden Zugrif auf die Datenbank muss diese Verbindung verwendet werden um einen Zugrif auf jeweilige Tabelle aufzubauen.

\begin{lstlisting}[language=Java]
Table table = connection.getTable(TableName.valueOf(tableName));
\end{lstlisting}

nach dem Abarbeiten der Anfrage muss der Zugrif auf die Tabelle wieder geschloßen werden.
\begin{lstlisting}[language=Java]
table.close();
\end{lstlisting}

Sollte der Server beendet werden so soll auch die Verbindung zur Datenbank auch geschloßen werden.
Um das umzusetzen ist es sinnvoll das Schließen der Verbindung zur Datenbank in die Methode mit der Annotation     @PreDestroy zu implementieren.

\begin{lstlisting}[language=Java]
@PreDestroy
public void closeConnection() throws IOException {
       connection.close();
}
\end{lstlisting}

